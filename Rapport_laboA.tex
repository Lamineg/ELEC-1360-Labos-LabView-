\documentclass[11pt]{article}
\usepackage[utf8]{inputenc}
\usepackage[T1]{fontenc}
\usepackage[french]{babel}
\usepackage{amsmath}
\usepackage[bookmarks={true},bookmarksopen={true}]{hyperref}
\usepackage{graphicx}
\usepackage[a4paper]{geometry}
\pagestyle{plain}
\setlength{\parindent}{5mm}
\usepackage{amsmath}
\usepackage{color}
\usepackage{caption} 

\title{\textbf{LFSAB1508\\ Projet P4 - Electricité \\ MS1} \\ {\large Groupe 27}}
\author{Guillaume \bsc{Lamine} \\(7109-13-00) \and Guy \bsc{Mavungu Zola Lutete} \\()  \and Emmanuel \bsc{Dushimimana} \\ (8617-11-00)}
\date{date}
\date{\vspace*{25mm}
\includegraphics[scale=0.75]{logo.jpg}\\
		\vspace*{30mm} 
		\begin{center}
		Année académique 2015-2016 \\	
		\end{center}}

\begin{document}
\thispagestyle{empty}

\maketitle
\thispagestyle{empty}
%\tableofcontents
\setcounter{tocdepth}{3}
\setcounter{page}{1}
\newpage


\section*{Introduction}
Le laboratoire A nous a permis de prendre en main l'outil USRP. \\ Les sections suivantes sont le résumé de l'apprentissage acquis lors du laboratoire.

\section{le graphique du taux d'erreur(BER)}

 \begin{figure}[!h]
     \centering
     \includegraphics[width=10cm]{graphBER}
     \caption{le graphique du taux d'erreur (BER) vs $\frac{1}{N_0}$}
     \label{ber}
 \end{figure}
 
\section{Réponses aux questions du Pre Lab}
 
\subsection{What is the range of allowable carrier/center frequency supported by
the NI-USRP?}

Frenquency range : 50 MHz à 2.2 GHz

\subsection{What is the maximum allowable bandwidth supported by the NI-USRP?}

Maximum instantaneous real-time bandwidt : 20MHz pour une taille d'échantillon de 16 bits.

\subsection{What is the maximum sampling rate of the NI-USRP?}

25 MS/s pour une taille d'échantillon de 16 bits

\subsection{Why do you think the DDC is implemented? What is its main benefit?}

\subsection{In your own words, describe what the bandwidth of an instrument is.}

\subsection{What is meant by the sampling rate of an instrument?}

\subsection{Why are these specifications important for designing a transmitter and
receiver in a wireless communications system?}

\section{Reponses aux questions Lab1:Part2}

\subsection{What is the IP Address for your NI USRP?}
\subsection{Describe what happens to the spectral response of the transmitted signal
when you place an obstruction between the antenna of transmitter and
that of the receiver (i.e., blocking the line of sight component of the
waveform)}
\subsection{What are the three elements in the data cluster output from niUSRP
Fetch Rx Data (poly).vi? Give the name and data type of each element}

\end{document}
