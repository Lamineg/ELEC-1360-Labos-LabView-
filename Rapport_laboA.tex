\documentclass[11pt]{article}
\usepackage[utf8]{inputenc}
\usepackage[T1]{fontenc}
\usepackage[french]{babel}
\usepackage{amsmath}
\usepackage[bookmarks={true},bookmarksopen={true}]{hyperref}
\usepackage{graphicx}
\usepackage[a4paper]{geometry}
\pagestyle{plain}
\setlength{\parindent}{5mm}
\usepackage{amsmath}
\usepackage{color}
\usepackage{caption} 

\title{\textbf{LFSAB1508\\ Projet P4 - Electricité \\ MS1} \\ {\large Groupe 27}}
\author{Guillaume \bsc{Lamine} \\(7109-13-00) \and Guy \bsc{Mavungu Zola Lutete} \\()  \and Emmanuel \bsc{Dushimimana} \\ (8617-11-00)}
\date{date}
\date{\vspace*{25mm}
\includegraphics[scale=0.75]{logo.jpg}\\
		\vspace*{30mm} 
		\begin{center}
		Année académique 2015-2016 \\	
		\end{center}}

\begin{document}
\thispagestyle{empty}

\maketitle
\thispagestyle{empty}

%\tableofcontents
\setcounter{tocdepth}{3}
\setcounter{page}{1}
\newpage

\section*{Introduction}
Le laboratoire A nous a permis de prendre en main l'outil USRP. \\ Les sections suivantes sont le résumé de l'apprentissage acquis lors du laboratoire.
\section{le graphique du taux d'erreur(BER)}
 \begin{figure}[!h]
     \centering
     \includegraphics[width=10cm]{graphBER}
     \caption{le graphique du taux d'erreur (BER) vs $\frac{1}{N_0}$}
     \label{ber}
 \end{figure}
\section{}
\section{Reponses aux questions Lab1:Part2}
\subsection{}

\end{document}
