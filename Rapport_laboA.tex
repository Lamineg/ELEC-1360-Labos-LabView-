\documentclass[11pt]{article}
\usepackage[utf8]{inputenc}
\usepackage[T1]{fontenc}
\usepackage[french]{babel}
\usepackage{amsmath}
\usepackage[bookmarks={true},bookmarksopen={true}]{hyperref}
\usepackage{graphicx}
\usepackage[a4paper]{geometry}
\pagestyle{plain}
\setlength{\parindent}{5mm}
\usepackage{amsmath}
\usepackage{color}
\usepackage{caption} 

\title{\textbf{LFSAB1508\\ Projet P4 - Electricité \\ MS1} \\ {\large Groupe 27}}
\author{Guillaume \bsc{Lamine} \\(7109-13-00) \and Guy \bsc{Mavungu Zola Lutete} \\()  \and Emmanuel \bsc{Dushimimana} \\ (8617-11-00)}
\date{date}
\date{\vspace*{25mm}
\includegraphics[scale=0.75]{logo.jpg}\\
		\vspace*{30mm} 
		\begin{center}
		Année académique 2015-2016 \\	
		\end{center}}

\begin{document}
\thispagestyle{empty}

\maketitle
\thispagestyle{empty}
%\tableofcontents
\setcounter{tocdepth}{3}
\setcounter{page}{1}
\newpage


\section*{Introduction}
Le laboratoire A nous a permis de prendre en main l'outil USRP. \\ Les sections suivantes sont le résumé de l'apprentissage acquis lors du laboratoire.

\section{le graphique du taux d'erreur(BER)}

 \begin{figure}[!h]
     \centering
     \includegraphics[width=10cm]{graphBER}
     \caption{le graphique du taux d'erreur (BER) vs $\frac{1}{N_0}$}
     \label{ber}
 \end{figure}
 
\section{Réponses aux questions du Pre Lab}
 
\subsection{What is the range of allowable carrier/center frequency supported by
the NI-USRP?}

Frenquency range : 50 MHz à 2.2 GHz

\subsection{What is the maximum allowable bandwidth supported by the NI-USRP?}

Maximum instantaneous real-time bandwidth : 20MHz pour une taille d'échantillon de 16 bits.

\subsection{What is the maximum sampling rate of the NI-USRP?}

25 MS/s pour une taille d'échantillon de 16 bits

\subsection{Why do you think the DDC is implemented? What is its main benefit?}

Le digital down converter (DDC), permet d'effectuer une décimation dans le "Hardware" au lieu de le faire dans le software, ce qui permet d'être plus rapide. La décimation est le processus de réduire le taux d'échantillonage d'un signal déjà échantilloné. Cela va donc réduire la mémoire nécessaire pour stocker le signal.

\subsection{In your own words, describe what the bandwidth of an instrument is.}

La largeur de bande d'un instrument est la bande qui couvre l'ensemble des fréquences avec lesquelles le USRP travaille, c'est-à-dire celle sur laquelle le signal va être transmi. Chaque instrument, ici le USRP, va travailler dans une bande qui lui a été prédéfinie.

\subsection{What is meant by the sampling rate of an instrument?}

Le taux d'échantillonage est le nombre d'échantillons, qui sont fait par seconde lors de l'échantillonage d'un signal du domaine continu vers le domaine en temps discret.

\subsection{Why are these specifications important for designing a transmitter and
receiver in a wireless communications system?}

Ces spécifications sont indispensables car elles vont varier selon les applications. Elles doivent être connues afin de permettre la conception et l'utilisation adéquate de l'instrument.
En effet, d'abord, chaque instrument occupe en général un pannel de fréquences bien précises qui lui sont propres afin de ne pas interferer avec des signaux non attendus. De plus, elles sont nécessaire pour vérifier que la fréquence d'échantillonage est au moins supérieure à deux fois la fréquence de la largeur de bande (Théorème de Nyquist).

\section{Reponses aux questions Lab1:Part2}

\subsection{What is the IP Address for your NI USRP?}
Les adresses IP du NI USRP  sont listées dans le tableau ci-dessous. On peut y lire l'adresse IP du transmetteurs et du receveur. \\ \\
 \begin{tabular}{|c|c|}
      \hline
      TX & RX  \\
      \hline
      192.168.10.51 & 192.168.52 \\
      \hline
 \end{tabular}
\subsection{Describe what happens to the spectral response of the transmitted signal
when you place an obstruction between the antenna of transmitter and
that of the receiver (i.e., blocking the line of sight component of the
waveform)}
Cette expérience n'a pas pu être réaliser puisque les antennes ne sont pas accessibles. 
\subsection{What are the three elements in the data cluster output from niUSRP
Fetch Rx Data (poly).vi? Give the name and data type of each element.}
data returns the received baseband samples as complex, double-precision floating-point data in a waveform data type, which also includes sampling information.
\\
\begin{itemize}
 \item t0(data type:dbl) specifies the trigger (start) time of the acquired Y array
\item dt(date type:dbl) specifies the time between values in the Y array. 
\item Y(data type:dbl) specifies the complex-valued baseband waveform. The real and imaginary parts of this complex data           array correspond to the in-phase (I) and quadrature-phase (Q) data, respectively. 
\end{itemize}
\subsection{In RXRF config.vi the IQ rate parameter is used by niUSRP Configure
Signal.vi. Can theUSRPsupport any arbitrary IQ rate? If not, what will
the USRP do when an unsupported bandwidth parameter is requested?}

Non car l'USRP est configué pour surporter un certain taux de IQ. Alors on lui passe un taux de IQ arbitraire, ce dernier est remplacé par une valeur proche de valeurs existantes.

\end{document}
